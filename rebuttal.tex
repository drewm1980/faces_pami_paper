\documentclass[11pt]{article}

\usepackage{epsfig}
\usepackage{latexsym, amsmath, amsfonts}
%\usepackage{pdfsync}

\renewcommand{\baselinestretch}{1.0}
\parskip 2.2mm
\parindent 0mm
\topmargin -0.60in \oddsidemargin 0.0625in \textheight 9.00in
\textwidth 6.50in

\renewcommand{\Re}{{\mathbb R}}
\newcommand{\Ze}{{\mathbb Z}}
\def\QED{~\rule[-1pt]{5pt}{5pt}\par\medskip}
\newenvironment{proof}{{\bf Proof: \ }}{ \hfill \QED}

\newcommand{\ie}{{\it i.e., }}
\newcommand{\D}{\displaystyle}
\newtheorem{lemma}{Lemma}
\newtheorem{theorem}{Theorem}
\newtheorem{fact}{Fact}
\newtheorem{corollary}{Corollary}
\newtheorem{remark}{Remark}
\newtheorem{assumption}{Assumption}
\newtheorem{definition}{Definition}



\begin{document}

   \bibliographystyle{plain}

   \title{\Large {\bf Manuscript  Revision Statement}}

       \author{{Andrew Wagner, John Wright, Arvind Ganesh, Zihan Zhou, Hossein Mobahi, and Yi Ma}}

   \date{}
   \maketitle

We thank the reviewers for their constructive comments, and in particular
reviewer 1 for an extra in-depth review.  While we have philosophical
differences on several points, we deeply appreciate that the valuable constructive 
criticisms raised by the reviewers have helped us strengthen the paper considerably.  
Below we will respond to comments of the reviewers in order as well as indicate 
specific changes that we have made to address their concerns.


\section{Respond to Reviewer \#1}
Reviewer 1 makes several specific recommendations for improving the
presentation of the paper, many of which we have heeded.  Please refer to the change
list for details. 

\paragraph{About Presentation:} The reviewer has suggested we can reduce the description
of the ALM algorithm in Section II-D. We agree ALM has been extensively studied in the optimization 
literature. We do not claim any novelty here. However, we believe a complete description of the algorithm 
here is important for the TPAMI audience: There are
many algorithms for solving the associated optimization problem, with drastically different complexity
and speed. Typical vision researchers might not be so familiar with some of the recent
developments in these algorithms. In particular some of the more effective improvements to ALM are fairly 
recent and scattered in different literatures. In fact, many of the questions raised by reviewer \#2
ask for even more detailed explanation about the algorithms. So we believe, a complete but brief 
description of the ALM-based algorithm is important for the audience of TPAMI to implement and 
test the ideas efficiently and correctly.

We find other suggestions on presentation useful and have made changes accordingly. 

\paragraph{About Positioning and Novelty:}
As we have tried to make it clear in the very beginning, the proposed method does not aim to solve face 
recognition for all scenarios! It is not designed for the case where the training is uncontrolled and limited, hence it
may not have any advantages over other face recognition techniques/systems specifically designed for such scenarios. 
However, there are many practical scenarios where the proposed setting is very relevant: access control
or recognizing refugees, where cooperative training images can (and should) be obtained in advance but no cooperative test
images are needed or available. No need of cooperative test images is arguably the main (or only) advantage that 
face recognition has over other biometric measures for security, such as fingerprint, iris, and DNA. 
If the subjects can cooperate for both training and testing, as suggested by the reviewers, then face 
recognition may no longer be the best choice, in terms of cost and accuracy. The proposed system is an attempt
to build a system that is not only accurate and robust enough to compete with other biometric measures but also 
discreet and non-intrusive when in use. To the best of our knowledge (probably the reviewer may agree), even 
for this admittedly restricted scenario, there is no practical face-recognition system whose performance can meet
the standards of most access control and security purposes. Existing systems have found success  mostly in applications
that are less-demanding in performance such as online image search or photo album organization etc. 

\paragraph{About Summary Recommendation:} While we agree with the reviewer that careful evaluation of our
system is needed by comparing it with other face recognition techniques/systems. We must point out that most face recognition systems are not designed for the scenario considered by this paper, hence a fair and meaningful comparison can be difficult. Nevertheless, to better evaluate both the strength and limitation of our method, we have added experimental comparison of our system with LBP, one of the leading techniques used for face verification for most commercial face recognition systems (in addition to other techniques already compared in the paper). We compare both methods with the same settings on two datasets that are representative of controlled and uncontrolled training: the Multi-PIE and the FERET
datasets.  The results indicate both suboptimal performance of LBP for applications where many controlled training images (under different illuminations) are available, and conversely, suboptimal performance of our algorithm on the FERET dataset where the training is not controlled nor sufficient.

\paragraph{Detailed Concerns or Comments by Reviewer 1:}
\begin{enumerate}
\item As we have mentioned earlier, we completely agree with the reviewer that if the users are completely cooperative at
both training and testing phases, face recognition may not necessarily be the best choice of biometric measure -- fingerprint
and iris are arguably better choices. However, our method does NOT require the users to be cooperative at the test
stage. The purpose of having a good training set and a robust recognition system is precisely to alleviate the users 
from being so cooperative at the test stage. This will allow the system to be less intrusive and much more user-friendly 
as it can handle typically daily change of faces such as occlusion by glasses, hats, or beard and variation in lighting and shadow. Ideally, face recognition will take place automatically and unknowingly as a user walks through a door or works in front a desktop.

In addition, if the accuracy and reliability of a face recognition system can truly get close to that of fingerprint and iris, it can serve as a complementary measure in a multi-modal system and improve the overall security for high-stake applications. To the best of our knowledge, no existing face recognition system, with or without cooperative training, comes close to meet the standards of high-stake security applications. 

Furthermore, we have found that the use of the L1-norm as a robust error function improves performance 
even when the test image does not contain occlusions.  Any boundaries where the intensity changes in a
misaligned image can result in a large localised error component.  While
unoccluded faces are mostly smooth, sharp boundaries still occur where there
are cast shadows, especially at depth discontinuities such as the nostrils, the
sides of the nose, and partially open mouths.  


\item Reviewer 1's interest in face recognition for terrorist watchlist and law
enforcement applications is of limited relevance to the evaluation of this
article.  By claiming that ``the field of face recognition has set itself the
goal of recognition with as few gallery images as possible,'' the reviewer is
conflating the constraints of one particular type of applications of face recognition
(surveillance) with those of face recognition as a whole. This is a (noble) goal probably easy to
set than to meet. Although progresses have been made towards this goal, the reviewer
probably has to agree that after several decades (and several high-profile failures of
face recognition systems in real-world testing), people are still looking for techniques 
that can achieve the performance required by high-end applications such as homeland 
security. We believe that, before such techniques can be found or such a goal can be
achieved at all, we should not preventing ourselves from making progress in developing systems that 
can already provide good guarantee for success, even if the conditions are somewhat limited. 

A huge part of the message of our paper is that for applications that allow it, we can and should
be gathering more and higher quality data at enrollment to our advantage. In fact, majority 
of the crime watch-list consists of criminals who have been committed before. Today,
the acquisition standards for fingerprints are far more strict than that for
capturing face images of criminals and foreigners -- precisely because they are shown to
help fingerprint identification systems. Of course, the idea of better training helps is certainly
not new or even recent.  What is new is the confluence of: 1. The availability
of hardware to efficiently acquire such data (high intensity, high contrast
ratio DLP projectors), 2. Robust and efficient computational tools to use such data in a
principled fashion, 3. Computer hardware fast enough to even contemplate using
entire images as features.


\item ``It is conceivable that SRC is applicable to any feature as well - which
underscores the fact that it is simply a pattern classification toolbox - on
par with any found in a textbook on pattern recognition. What is it that makes
it special for face recognition?''

Most general pattern classification algorithms do not take into account the
physically motivated linear model that is central (though by no means unique)
to the problem of illumination variation.

\item Our goal is not to achieve the best recognition rates on the NIST family of
benchmarks (FERET, FRVT, FRGC, MBGC, GBU); rather it is to advance state of
the art in face recognition given cooperative subjects.  The motivating
application of the NIST data sets is government law enforcement (originally for
enforcement of drug prohibition, and more recently shifting towards border
security).  This requires backwards compatibility with old image data, some of
which was gathered before computers and digital cameras were even in widespread use.  
Due to these historical reasons, none of these data sets contain multiple controlled
illuminations.  For example, FRGC contains at most two illuminations per person
in the gallery.  (Two studio lights were kept on, and a third light was
switched either on or off.) The techniques used to gather these data sets are not a wise 
choice for applications where users are cooperative (foreigners applying for a visa)
or forced to be so (suspects or criminals in custody), and are needlessly
restrictive.  While large public data sets are critical to the rigorous
comparison of recognition algorithms, their existence should not be used to
argue against the development of new data acquisition techniques and algorithms
appropriate for them.

\item We have added comparison with a Local Binary Patterns based recognition system, and
have shown that both our system and LBP perform better in the application domains or settings
for which they were designed.  Please see the change list for details. 

\item Although we claim to have a preliminary system that works in a realistic
access control setting, we have never claimed that our method solves all problems in face
recognition. We do not know where the reviewer gets that wrong impression. 
From the very beginning and then at multiple places later, we actually emphasize
the importance of having the more restrictive training data for guaranteeing the success 
of our method. 

If it helps, we will try to tone down the rhetoric in XYZ places. UNFINISHED!!

\item We will add further discussion contrasting the advantages and disadvantages of our
system vs. systems incorporating 3D models. This paper does not intend to deal with large 3D pose
although the proposed techniques are directly applicable to the cases when images of more poses or 
even a full 3D models are available. 

\item The issues of (mis)alignment and illumination variance are coupled in the
sense that if one problem is solved, the other problem becomes a lot easier. But solving them
separately is not sufficient for solving the face recognition problem. 

\item If the data fits our model perfectly, at the correct alignment the only
corruptions are due to occlusions.  Until the images are perfectly aligned,
there is indeed a component of dense error caused by misalignment.  This dense
error component is indeed largest where the image gradient is high (or even
discontinuous), and the derivative of this error component is what drives the
algorithm to its solution.  We explicitly do blur and downsample the images to
mollify registration effects; multiscale alignment is a well-known technique,
and is not new to our paper.  

\item We have removed the reference to scattered light, as there appear to
be several common uses of the term.  That said, it is true that linear illumination
models are physically motivated for scenes far more general than for convex 
lambertian objects.  Maxwell's equations in free space are linear, and materials
that create non-linear boundary conditions are the exception
rather than the rule.  Without proportionality between radiance and irradiance 
the concept of a Bidirectional Reflectance Distribution Function and its
generalizations are not even meaningful. 

\item Rigorous justification of the choice of illuminations is a subject of 
continuing research.  Our current choice is motivated empirically by the
experiments described in the paper. 

\item To our knowlege, we are the first to use projectors for illumination in
face recognition applications.  Section 1A of the previous submission already
cites several other applications that use projectors as illumination sources.
The requested references propose a technique for reducing the SNR of projector
based illumination sysems for certain classes of objects.  We have added
discussion of this to the paper.

\item  It is true that one or both stages of the algorithm could be substituted
with other techniques.  That said, the registration and recognition stages are
coherent in the sense that they are motivated by same modeling assumptions
(illumination model, sparse occlusion, etc.), they use the same robust penalty
on image error (the L1-norm), and they use the same numerical optimization
techniques (ALM).  While conceptual simplicity is a secondary concern in
algorithm design, starting with a minimalist foundation benefits both
practitioners as well as researchers.  Other researchers have already published
more complex extensions to our work. 

\item Since we perform alignment separately for each class in the gallery, the
computation for the alignment stage is linear in the size of the gallery.  The
variation in alignment across classes is indeed small, but still changes the
objective enough to affect recognition results.  Aligning to a generic face (or
a gallery based on ``aligned'' images of many people) is a good idea, and one we
have considered as well.  If nothing else, it could be used to improve
alignment initialization instead of using raw face detector output.  It may
also be useful to use the roughly aligned test image to quicky eliminate some
of the classes from further consideration.  Again, we have deliberately tried
to keep the algorithm as minimalist as possible for this publication.

\item The adaptation of the Lucas-Kanade algorithm to both the L1 norm cost
function and a linear illumination model is novel to our knowlege.  As
discussed earlier, we disagree with Reviewer 1's contention that data
acquisition is not a critical part of the face recognition problem -- careful (training) 
data acquisition has been a critical part for almost all other biometric identification 
systems (e.g. fingerprint and iris and DNA)! Why face recognition needs to be so different? 

We believe we have demonstrated that using a well-though-out combination of existing ideas
(iterative image alignment, L1-error function, SRC, using projectors for
illumination), one can build a recognition system that is conceptually simple, well
motivated, and competitive with state-of-the-art recognition systems for access
control scenarios.

\end{enumerate}

\section{Respond to Reviewer \#2}
Reviewer 2 raises 5 questions, which we will now address.  Since the
recognition optimization is convex and has no difficultly converging, we assume
that (1)-(3) are asking about the iterative alignment scheme.

\begin{enumerate}
\item (i) ``How often an infinite loop is encountered, and how it is handled."

We place a hard limit on the maximal number of alignment iterations, so the
algorithm is guaranteed to terminate in finite time.  In practice, the
alignment algorithm tends to converge within 4-10 iterations for each
resolution of the multi-scale implementation.  We also place hard bounds on how
large the alignment can be, and terminate alignment early if these bounds are
reached.  In the cases where the alignment ``slips off'' the face, this is how
alignment terminates.  

\item (ii) ``How often the starting point happens to be located in a wrong basin of
attraction, i.e. a wrong face. (If there is any way to know this.)''


Since we align the test image to the training users individually, we avoid the
potential problem of the test image converging to a local minimum corresponding
to a non-match face.

\item (iii) ``How often the solution gets drifted away into a wrong basin of
attraction. (If there is any way to know this.)''

In Section II E, we have conducted careful experiments to measure the region of attraction
for the proposed alignment method. 

\item (iv) ``What is the criteria or threshold for convergence? If the threshold is
stringent, the algorithms are likely to get stuck in infinite loops, and if the
threshold is liberal, premature convergence would occur. How are such
thresholds picked and how universal are they? Does every new dataset need its
own threshold?''

For the l1 minimizations we threshold the l1-norm of the difference between
successive values of the state (this includes the both the explicit
optimization variables and the Largrange multipliers).  Similarly, for the
alignment iteration, we threshold the l1-norm of the update to the
transformation parameters.  We used the same set convergence thresholds for all
of the data sets we ran against.  With the suggested thresholds, the l1-minimzations are
guaranteed to converge eventually, although convergence may be very slow if they
are chosen poorly.  

\item (v) ``The problem, however, is that if we minimize a function one variable at
a time, we may actually end up increasing the function $(L_{\mu}(x,e,\lambda_{k}))$ value.''

This may be true for general nonlinear optimization problem. However for convex problems
we are solving here, the alternating direction methods have been proven to converge, see
[27] and related work by the same authors. 


\end{enumerate}

\section{Respond to Reviewer \#3}
Reviewer 3 makes the following requests:

\begin{enumerate}
\item ``I would have liked to see the experiment for the case S=1, without
impostors in the dataset, to judge the effect of the ``Sparse
Concentration Index''.

We have added experiments to the paper demonstrating the utility of the
sci-based recognition stage for both identification and imposter rejection.  In
general, we see a few percentage points of improvement in recognition rate when
using SCI over recognition based purely on alignment residual.  
It is worth emphasizing that while this effect is desirable, 
the primary value of SCI is its use for imposter rejection. 

\item ``I would also retitle the paper to include ``robust measure''
instead of ``sparse representation", as the L1 Cost is just a robust
cost, and the system would probably work as well or better with other
(even non-convex) robust cost functions.''

We hope reviewer will find our supplemental justification for the (Sparse
Representation based) recognition convincing.  If so, the reviewer may
therefore also agree that the current title is indeed appropriate.

\item ``It might be possible to streamline the introduction a bit more, such that it
clearly describes the algorithm before going into the details.''

We have attempted to make the overall structure of our system clear earlier in the
paper, by clarifying early on that SRC forms the recognition stage of our system
in the introduction, as well as by making it clear in the caption of the motivating example
is not the final pipeline.

\item ``Using and requiring several input images of a person asks for a
comment on the advantage or difference to a system that uses
a 3D representation of a face. Please comment on this!"

The primary advantage of systems that internally use a 3D model of human faces
is that they have the potential to work over a wider range of pose variation in
the test image.  In general, systems using 2D models are simpler and easier to
engineer.  As far as data acquisition is concerned, 2D cameras still have a far
higher resolution and lower noise levels than 3D cameras.  Based on the principles
of photometric stereos, our set of 2D images may very well encode more useful shape 
information about a user's face than a single low-resolution depth image, especially if 
recognition is to be performed on 2D input images.

\end{enumerate}

\end{document}
